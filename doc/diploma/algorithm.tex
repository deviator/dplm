\subsection{Реализация решения задач МБПЛА}

В данной работе основной рассматриваемой ситуацией является обследование
урбанистической местности без построения карты помещений. Это в первую
очередь связанно с ограничениями аппаратуры, на которой производилась 
симуляция. Конкретно это связанно с обграничением на объём видеопамяти,
так как именно там хранится карта. Обойти это ограничение можно, реализовав
алгоритм распределённого хранения данных в массиве на уровне ПМО. Но это
значительно усложняет архитектуру и не влияет на работу других алгоритмов.

В работе реализован алгоритм перемещения юнитов для построения карты
местности и поддержания её актуальности. В процессе реализации алгоритм
разбит на несколько частей, для некоторых частей существуют вариации.

Шаг алгоритма работы юнита вызывается с помощью класса \verb|Model|.
Также класс \verb|Model| передаёт в юнит координаты и скорости других
юнитов, как опастные точки, которые юнит должен обходить.

Алгоритм работы юнита в общих чертах:

\begin{figure}[h!]

    \centering

    \begin{tikzpicture}[->,=>stealth',shorten >=0pt,auto,
            node distance=1.2cm, thick,
        startnode/.style={circle,minimum size=.5cm,fill=black},
        stepnode/.style={draw,rectangle, minimum height=0.7cm,
        text centered, rounded corners},
        endnode/.style={circle,minimum size=.5cm,fill=white,draw=black} ]

        \node[startnode] (0) {};
        \node[stepnode] (1) [below of=0] {$\text{отработка логики}$};
        \node[stepnode] (2) [below of=1] {$\text{реализация перемещения}$};
        \node[stepnode] (3) [below of=2] {$\text{служебные действия}$};
        \node[endnode] (4) [below of=3] {};

        \path
        (0) edge (1) 
        (1) edge (2) 
        (2) edge (3)
        (3) edge (4);

    \end{tikzpicture}

    \caption{Шаг работы юнита}
    \label{fig:unit_algo_min}
\end{figure}

К служебным действиям относятся:
\begin{mintemize}
\item пересчёт матрицы трансформации
\item запись текущего положения в историю перемещения
\end{mintemize}

\newpage
\subsubsection{Отработка логики}

Для класса \verb|Unit| логика определена частично. Своё продолжение
она находит в производных классах (см. рис. \nnref{fig:model_uml} на стр. \pageref{fig:model_uml}).

\tikzstyle{startnode} = [circle,minimum size=.5cm,fill=black]
\tikzstyle{stepnode} = [draw,rectangle, minimum height=0.7cm,
            text centered, rounded corners]
\tikzstyle{endnode} = [circle,minimum size=.5cm,fill=white,draw=black]
\tikzstyle{statnode} = [diamond,aspect=3,draw,text badly centered,inner sep=0pt]
\tikzstyle{every picture} = [->,=>latex,shorten >=0pt,auto,node distance=1.5cm,thick]

\begin{figure}[h!]

    \centering

    \begin{tikzpicture}

        \node[startnode] (0) {};
        \node[stepnode] (1) [below of=0] {$\text{обновление списка ближайших точек}$};
        \node[stepnode] (2) [below of=1] {$\text{вызов расчёта путевой точки}$};
        \node[stepnode] (3) [below of=2] {$\text{вызов расчёта направления камеры}$};
        \node[endnode] (4) [below of=3] {};

        \path
        (0) edge (1) 
        (1) edge (2) 
        (2) edge (3)
        (3) edge (4);

    \end{tikzpicture}

    \caption{Логика Unit}
    \label{fig:unit_algo_logic}
\end{figure}

Функция расчёта путевой точки из себя представляет присвоение значения поля \verb|vec3 target|
полю \verb|vec3 wayPoint|. Камера направляется по вектору текущей скорости.

Ниже по иерархии наследования стоит класс \verb|AutoTargetUnit|. В нём реализован
алгоритм, отслеживающий прогресс перемещения юнита к целевой точке. Этот алгоритм
заключён в функции \verb|retargetLogic|. Смысл его состоит в том, что когда юнит
не может обойти препятствие, дисперсия последних значений положения юнита не
будет выходить за пределы установленного порога. Простыми словами, этот
алгоритм отрабатывает в случае, если юнит "топчется на месте". Для получения
последних положений юнита используется массив \verb|hist|.

\phantomsection
\label{ref:unit_algo_retarget}

На рис. \ref{fig:unit_algo_retarget} представлен алгоритм работы
функции \verb|retargetLogic|, где:

$n$ -- счётчик вызовов

$K$ -- количество последних положений для проверки

\verb|bool needNewTarget()| -- функция, сравнивающая дисперсию с минимальным порогом

\verb|void choiseTarget()| -- функция, реализующая выбор следующей целевой точки.

Функция \verb|void choiseTarget()| реализуется в наследниках \verb|AutoTargetUnit|.

\begin{figure}[h!]

    \centering

    \begin{tikzpicture}

        \node[startnode] (0) {};
        \node[stepnode] (1) [below of=0] {$n++$};
        \node[statnode] (2) [below of=1] {$n<K$};
        \node[endnode] (r1) [right=2cm of 2] {};
        \node[statnode] (3) [below of=2] {$!needNewTarget()$};
        \node[endnode] (r2) [below of=r1] {};
        \node[stepnode] (4) [below of=3] {$n=0$};
        \node[stepnode] (5) [below of=4] {$choiseTarget()$};
        \node[endnode] (6) [below of=5] {};

        \path
        (0) edge (1) 
        (1) edge (2) 
        (2) edge (3)
        (3) edge (4)
        (2) edge node[above] {$\text{да}$} (r1)
        (3) edge node[above] {$\text{да}$} (r2)
        (4) edge (5)
        (5) edge (6);

    \end{tikzpicture}

    \caption{Алгоритм функции retargetLogic}
    \label{fig:unit_algo_retarget}
\end{figure}

Класс \verb|RndTargetUnit| реализует в функции \verb|choiseTarget| случайный
выбор целевой точки.

Класс \verb|FindTargetUnit| реализует функцию \verb|choiseTarget| по алгоритму на
рисунке \ref{fig:unit_algo_find_choise}.
\phantomsection
\label{ref:algo:choise:find}

\begin{figure}[h!]

    \centering

    \begin{tikzpicture}

        \node[startnode] (0) {};
        \node[stepnode] (1) [below of=0] {$\text{выбор случайного угла}$};
        \node[stepnode] (2) [below of=1] {$\text{получение информации о карте в 4х регионов}$};
        \node[stepnode] (3) [below of=2] {$\text{подсчёт количества неизвестных точек и их центра в регионах}$};
        \node[stepnode] (4) [below of=3] {$\text{выбор региона с большим содержанием неизвестных точек}$};
        \node[stepnode] (last) [below of=4] {$\text{выставление целевой точки в центр неизвестных точек в выбранном регионе}$};
        \node[endnode] (end) [below of=last] {};

        \path
        (0) edge (1) 
        (1) edge (2) 
        (2) edge (3)
        (3) edge (4)
        (4) edge (last)
        (last) edge (end);

    \end{tikzpicture}

    \caption{Алгоритм функции choiseTarget для класса FindTargetUnit}
    \label{fig:unit_algo_find_choise}
\end{figure}

Пункт <<получение информации о карте в 4х регионах>> отрабатывается так:

$$ R_i = findTargetAround( pos + rotate( vec2(1,0), angle + \frac{(i-1)\cdot\pi}{2} ) )$$

где:

$i \in [1,4]$ -- номер региона,

$pos$ -- текущее положение юнита,

$angle$ -- случайный угол (не изменяется, после вычисления в начале функции),

$rotate$ -- функция поворота вектора,

$findTargetAround$ -- получение информации вокруг заданной точки,

$R_i$ -- информация о регионе, содержит набор секторов карты.

Внутри функции \verb|findTargetAround| происходит вычисление региона
вокруг точки и обращение к карте, вычисление центра неизвестных точек и
подсчёт их количества.

\phantomsection
\label{ref:algo:choise:serial}

Последнюю вариацию алгоритма \verb|choiseTarget| реализует класс
\verb|SerialTargetUnit|. В этой вариации алгоритма выбора целевой точки
карта разбивается на равные регионы, в которые отправляются группы юнитов
для исследования и патрулирования. Для работы этой вариации алгоритма юниту
необходимо знать:

\begin{mintemize}
\item размер карты
\item количество юнитов в системе
\item порядковый номер юнита в массиве
\end{mintemize}

Номер региона для юнита вычисляется следующим образом:

$$N = K \% ( Mr_x \cdot Mr_y \cdot Mr_z ) $$

где:

$N$ -- номер региона,

$K$ -- номер юнита,

$Mr_{[x,y,z]}$ -- количество регионов карты по соответствующим осям,

$\%$ -- операция взятия остатка от деления.

Номер начальной целевой точки юнита вычисляется следующим образом:

$$T = K / ( Mr_x \cdot Mr_y \cdot Mr_z )$$

где:

$T$ -- номер точки,

$\%$ -- операция целочисленного деления.

В самой функции \verb|choiseTarget| происходит инкрементирование к
значению номера текущей точки количества юнитов на данный регион

$$T = T + L / ( Mr_x \cdot Mr_y \cdot Mr_z )$$

где:

$L$ -- общее количество юнитов в системе.

Для правильной работы необходимо, чтобы количество юнитов системы $F$ превышало
количество регионов карты.

Вычисление положения целевой точки происходит по следующей формуле:

$$ \vec target = \vec offset + getCoord( \vec Mr, N ) \cdot \vec N_{cell} + ( getCoord( Ir, T ) + vec3(0.5) ) \cdot \vec T_{cell} $$

где:

$\vec A \cdot \vec B$ -- умножение векторов производится покомпонентно,

$target$ -- координата целевой точки,

$offset$ -- смещение начала системы координат карты относительно основной
    системы координат,

$Mr$ -- размер карты в количестве регионов,

$N_{cell}$ -- размер региона в основной системе координат,

$Ir$ -- размер региона в количестве секторов,

$T_{cell}$ -- размер ячейки региона в основной системе координат,

$getCoord$ -- вектор-функция, получающая на входе размер сетки и индекс в ней, возвращающая координату,
    соответствующую этому индексу.

\newpage
\subsubsection{Отработка перемещения}

Отработка дифференциального уравнения реализована в классе \verb|BaseUnit|, 
от которого наследуется \verb|Unit| (см. \nnref{sec:unit_math_model} на стр. \pageref{sec:unit_math_model}).
Там же расчитывается сила сопротивления воздуха:

$$\vec F_D = -\vec V \cdot |V| \cdot C_{x0} \cdot S \cdot \rho \cdot 0.5$$

где:

$\vec F_D$ -- сила сопротивления воздуха,

$\vec V$ -- скорость юнита,

$C_{x0} \cdot S$ -- в программе \verb|CxS| -- произведение площади на коэффициент сопротивления,

$\rho$ -- плотность воздуха, в программе равна 1.

В классе \verb|Unit| реализуется функция расчёта управляющего воздействия.
Управляющее воздействие вычисляется как сумма

\begin{mintemize}
\item управляющего воздействия по целевой точке
\item корректирующего воздействия по ближайшим опасным точкам
\end{mintemize}

Сумма ограничивается функцией \verb|limitedForce|, которая реализована в
классе \verb|BaseUnit| (см. \pageref{ref:unit_model_limited_force}).

\newpage
\textbf{Корректирующее воздействие по ближайшим опасным точкам}

Важно, чтобы юниты не сталкивались друг с другом и не врезались в стены.
Реализация коррекции движения производится через аддитивное управляющее
воздействие $f_{\text{к}}$. 

Алгоритм вычисления $f_{\text{к}}$:

\begin{figure}[h!]

    \centering

    \begin{tikzpicture}

        \node[startnode] (0) {};
        \node[stepnode] (1) [below of=0] {$\text{координаты заполненных точки карты совмещают в массив с координатами юнитов}$};
        \node[stepnode] (2) [below of=1] {$\text{выбираются только те точки, что ближе к юниту чем} D_{min}$};
        \node[stepnode] (3) [below of=2] {$\text{для каждой точки вычисляетются 2 составляющие} f_{\text{к}}: \text{нормальная и тангенсальная}$};
        \node[stepnode] (last) [below of=3] {$\text{складываются результаты}$};
        \node[endnode] (end) [below of=last] {};

        \path
        (0) edge (1) 
        (1) edge (2) 
        (2) edge (3)
        (3) edge (last)
        (last) edge (end);

    \end{tikzpicture}

    \caption{Алгоритм функции nearCorrect}
    \label{fig:unit_algo_near_correct}
\end{figure}

Для наглядного представления алгоритма корректировки приведём векторную
схему.

\begin{figure}[h!]

    \centering

    \tikzstyle{every picture}+=[
        axis/.style={black},
        vector/.style={->,thick,black},
        help/.style={gray,thin,dashed},
        rot/.style={tdplot_rotated_coords}
    ]

    \def\dot{circle[radius=0.5mm]}
    \def\P{1.9}
    \def\Pm{1.6}
    \def\Pmm{0.5}
    \def\Vrot{20}
    \def\fnlen{5}
    \def\ftlen{2}
    \def\dst{8}

    \tdplotsetmaincoords{0}{0}
    \begin{tikzpicture}[tdplot_main_coords]
        \tdplotsetrotatedcoords{85}{60}{-50}
        \fill[rot,red] (\dst,  0,  0) \dot;
        \fill[rot,red] (\dst, .5,  0) \dot;
        \fill[rot,red] (\dst,-.5,  0) \dot;
        \fill[rot,red] (\dst, .5, .5) \dot;
        \fill[rot,red] (\dst,-.5, .5) \dot;
        \fill[rot,red] (\dst,  0, .5) \dot;
        \fill[rot,red] (\dst,  0,-.5) \dot;
        \fill[rot,red] (\dst, .5,-.5) \dot;
        \fill[rot,red] (\dst,-.5,-.5) \dot;

        \fill[rot,black] (0,0,0) \dot node[below] {$O$};

        \draw[rot,vector] (0,0,0) -- (\dst,0,0);
        \node[rot,above] at (6,0,0) {$\vec D$};
        \draw[rot,vector] (0,0,0) -- (1,0,0);
        \node[rot,below] at (1,0,0) {$\vec e_D$};

        \draw[rot,vector] (0,0,0) -- (\Vrot:5) node[above] {$\vec V$};
        \draw[rot,vector] (0,0,0) -- (\Vrot:1) node[above] {$\vec e_V$};
        \draw[rot,vector,help] (0,0,0) -- (0,0,3);
        \node[rot,left] at (0,0,3) {$\vec U$};
        \draw[rot,vector,red] (0,0,0) -- (-\fnlen,0,0);
        \draw[rot,vector,red] (0,0,0) -- (-1,0,0);
        \node[rot,below left] at (-\fnlen,0,0) {$\vec f_N$};
        \node[rot,below] at (-1,0,0) {$\vec N$};
        \draw[rot,vector,green] (0,0,0) -- (0,\ftlen,0);
        \draw[rot,vector,green] (0,0,0) -- (0,1,0);
        \node[rot,above left] at (0,\ftlen,0) {$\vec f_T$};
        \node[rot,above] at (0,1,0) {$\vec T$};
        \draw[rot,vector,blue] (0,0,0) -- (-\fnlen,\ftlen,0);
        \node[rot,above left] at (-\fnlen,\ftlen,0) {$\vec f_{\text{к}}$};

        \draw[rot,help] (0,\ftlen,0) -- ++(-\fnlen,0,0) -- ++(0,-\ftlen,0);

        \draw[rot,help] (0,0,\P) -- ++(0:\P) -- ++(0,0,-\P);
        \draw[rot,help] (0,0,\P) -- ++(\Vrot:\P) -- ++(0,0,-\P);
        \draw[rot,help] (0,\Pm,0) -- ++(\Pm,0,0) -- ++(0,-\Pm,0);
        \draw[rot,help] (-\Pmm,0,0) -- ++(0,\Pmm,0) -- ++(\Pmm,0,0);
    \end{tikzpicture}

    \caption{векторная схема работы nearCorrect}
    \label{fig:unit_algo_near_correct_vec}
\end{figure}

Обозначения на рисунке \ref{unit_algo_near_correct_ve}:

$O$ -- положение юнита,

$\vec D$ -- расстояние до точки,

$\vec e_D = \frac{\vec D}{|\vec D|}$ -- еденичный вектор расстояния,

$\vec V$ -- скорость юнита,

$\vec e_V = \frac{\vec V}{|\vec V|}$ -- еденичный вектор скорости юнита,

$\vec U = \vec e_D \times \vec e_V$ -- промежуточный вектор,

$\vec T = normalize(\vec U \times \vec e_D)$ -- направление тангенсальной коррекции

$\vec N = -\vec e_D$ -- направление нормальной коррекции.

Коррекция $\vec f_{\text{к}}$ вычисляется по формуле:

$$\vec f_{\text{к}} = ( ( \vec N \cdot ( D_{min} - |D| )^2 \cdot K_1 + \vec T ) \cdot max( K_2, < \vec e_D, \vec e_V > ) ) \cdot K_3 $$

где:

$K_{[1,2,3]}$ -- эмпирически подобранные коэффициенты, 

$< \vec e_D, \vec e_V >$ -- скалярное произведение.

Коэффициент $(D_{min} - |D|)^2$ увеличивается квадратично,
при пересечии юнитом минимально допустимой дистанции. Это позволяет
резко остановить юнит при прямом сближении с опасной точкой.

Коэффициент $max(0, < \vec e_D, \vec e_V > )$ обращается в ноль в случае,
если юнит двигается от опасной точки, и становится равным единице в случае,
если юинт имеет направление ровно на опасную точку. Это позволяет уменьшить
отталкивание от стен при паралельном движении вдоль них. Коэффициент $K_2$ оставляет
минимальное отталкивание от стены и ускорение вдоль стены, даже в случае, если
юнит движется от стены. Направление по коррекции по скорости позволяет лучше
обходить препятсвтия, нежели в случае направления по вектору дальности до цели.

Коэффициент $K_3$ позволяет подавлять управляющее воздействие по целевой \lb точке.

Такой алгоритм корректировки позволяет юнитам облетать выпуклые препядствия.
В случае, если юнит входит в "угол" реализуется выбор новой целевой точки
(см. функцию \verb|retargetLogic| на стр. \pageref{ref:unit_algo_retarget}).
