\section{Реализация решения задач МБПЛА}

???

\newpage
\subsubsection{Логика перемещения юнитов (!)}

Для системы в целом ставится задача полностью исследовать объём, отражаемый в карте.
Из этого следует, что каждый юнит направляется к ближайшему неизведанному участку.

Указание маршрута следования юниту даётся через так называемые целевые точки.
Их расстановка высчитывается совместно для всего массива. Первая итерация расчёта
случайная. На следующих итерациях расстановка происходит по следующему алгоритму:
\begin{mintemize}
\item для фиксированных объёмов карты (прямоугольные области) вычисляется суммарное
    количество неизведанных точек
\item выбирается область с максимальным количеством неизвестных точек
\item проверяется назначен ли этот объём другому юниту:
    \begin{mintemize}
        \item если нет, целевая точка назначается в центр области
        \item если назначен, выбрать вторую (и далее), по количеству неизвестных точек область
    \end{mintemize}
\item при достижении целевой точки юнит расчитывает следующую целевую точку
\end{mintemize}

Размер прямоугольных областей карты выбирается, исходя из максимальной дальности работы
датчика глубины юнита, и составляет удвоенное её значение.

\newpage
\subsubsection{Алгоритм перемещения юнитов (!)}

Важно, чтобы юниты не сталкивались друг с другом и не врезались в стены.
На данный момент реализация коррекции движения производится через аддитивное управляющее
воздействие $f_{\text{к}}$. 

Алгоритм вычисления $f_{\text{к}}$:
\begin{mintemize}
\item координаты заполненных точки карты совмещают в массив с координатами юнитов
\item выбираются только те точки, что ближе к юниту чем $D_{min}$
\item для каждой точки вычисляетются 2 составляющие $f_{\text{к}}$: нормальная и тангенсальная
\item складываются результаты
\end{mintemize}

\tikzstyle{every picture}+=[
    axis/.style={black},
    vector/.style={->,thick,black},
    help/.style={gray,thin,dashed},
    rot/.style={tdplot_rotated_coords}
]

\begin{center}

    \def\dot{circle[radius=0.5mm]}
    \def\P{1.9}
    \def\Pm{1.6}
    \def\Pmm{0.5}
    \def\Vrot{20}
    \def\fnlen{5}
    \def\ftlen{2}
    \def\dst{8}

    \tdplotsetmaincoords{0}{0}
    \begin{tikzpicture}[tdplot_main_coords]
        \tdplotsetrotatedcoords{85}{60}{-50}
        \fill[rot,red] (\dst,  0,  0) \dot;
        \fill[rot,red] (\dst, .5,  0) \dot;
        \fill[rot,red] (\dst,-.5,  0) \dot;
        \fill[rot,red] (\dst, .5, .5) \dot;
        \fill[rot,red] (\dst,-.5, .5) \dot;
        \fill[rot,red] (\dst,  0, .5) \dot;
        \fill[rot,red] (\dst,  0,-.5) \dot;
        \fill[rot,red] (\dst, .5,-.5) \dot;
        \fill[rot,red] (\dst,-.5,-.5) \dot;

        \fill[rot,black] (0,0,0) \dot node[below] {$O$};

        \draw[rot,vector] (0,0,0) -- (\dst,0,0);
        \node[rot,above] at (6,0,0) {$\vec D$};
        \draw[rot,vector] (0,0,0) -- (1,0,0);
        \node[rot,below] at (1,0,0) {$\vec e_D$};

        \draw[rot,vector] (0,0,0) -- (\Vrot:5) node[above] {$\vec V$};
        \draw[rot,vector] (0,0,0) -- (\Vrot:1) node[above] {$\vec e_V$};
        \draw[rot,vector,help] (0,0,0) -- (0,0,3);
        \node[rot,left] at (0,0,3) {$\vec U$};
        \draw[rot,vector,red] (0,0,0) -- (-\fnlen,0,0);
        \draw[rot,vector,red] (0,0,0) -- (-1,0,0);
        \node[rot,below left] at (-\fnlen,0,0) {$\vec f_N$};
        \node[rot,below] at (-1,0,0) {$\vec N$};
        \draw[rot,vector,green] (0,0,0) -- (0,\ftlen,0);
        \draw[rot,vector,green] (0,0,0) -- (0,1,0);
        \node[rot,above left] at (0,\ftlen,0) {$\vec f_T$};
        \node[rot,above] at (0,1,0) {$\vec T$};
        \draw[rot,vector,blue] (0,0,0) -- (-\fnlen,\ftlen,0);
        \node[rot,above left] at (-\fnlen,\ftlen,0) {$\vec f_{\text{к}}$};

        \draw[rot,help] (0,\ftlen,0) -- ++(-\fnlen,0,0) -- ++(0,-\ftlen,0);

        \draw[rot,help] (0,0,\P) -- ++(0:\P) -- ++(0,0,-\P);
        \draw[rot,help] (0,0,\P) -- ++(\Vrot:\P) -- ++(0,0,-\P);
        \draw[rot,help] (0,\Pm,0) -- ++(\Pm,0,0) -- ++(0,-\Pm,0);
        \draw[rot,help] (-\Pmm,0,0) -- ++(0,\Pmm,0) -- ++(\Pmm,0,0);
    \end{tikzpicture}

\end{center}

На рисунке:

$O$ -- положение юнита,

$\vec D$ -- расстояние до точки,

$\vec e_D = \frac{\vec D}{|\vec D|}$ -- еденичный вектор расстояния,

$\vec V$ -- скорость юнита,

$\vec e_V = \frac{\vec V}{|\vec V|}$ -- еденичный вектор скорости юнита,

$\vec U = \vec e_D \times \vec e_V$ -- промежуточный вектор,

$\vec T = normalize(\vec U \times \vec e_D)$ -- направление тангенсальной коррекции

$\vec N = -\vec e_D$ -- направление нормальной коррекции.

\newpage
Коррекция $\vec f_{\text{к}}$ вычисляется по формуле:

$$\vec f_{\text{к}} = ( ( \vec N \cdot ( D_{min} - |D| )^2 \cdot K_1 + \vec T ) \cdot max( K_2, < \vec e_D, \vec e_V > ) ) \cdot K_3 $$

где:

$K_{[1,2,3]}$ -- эмпирически подобранные коэффициенты, 

$< \vec e_D, \vec e_V >$ -- скалярное произведение.

Коэффициент $(D_{min} - |D|)^2$ увеличивается квадратично,
при пересечии юнитом минимально допустимой дистанции. Это позволяет
резко остановить юнит при прямом сближении с опасной точкой.

Коэффициент $max(0, < \vec e_D, \vec e_V > )$ обращается в ноль в случае,
если юнит двигается от опасной точки, и становится равным единице в случае,
если юинт имеет направление ровно на опасную точку. Это позволяет уменьшить
отталкивание от стен при паралельном движении вдоль них. Коэффициент $K_2$ оставляет
минимальное отталкивание от стены и ускорение вдоль стены, даже в случае, если
юнит движется от стены. Направление по коррекции по скорости позволяет лучше
обходить препятсвтия, нежели в случае направления по вектору дальности до цели.

Коэффициент $K_3$ позволяет подавлять управляющее воздействие по целевой \lb точке.

