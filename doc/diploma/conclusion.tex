\section{Заключение}

В работе были рассмотрен алгоритм построения карты местности и
поддержания карты в актуальном состоянии. Принцип работы алгоритма
состоит в указании целевых точек для каждого юнита. Был разработан
алгоритм обхода выпуклых препядствий, основанный на определении
вектора коррекции по текущей скорости и положению опасных точек.
Было разработанно 3 вариации алгоритма:

\begin{mintemize}
\item выбор случаной целевой точки
\item поиск неизведанных регионов вокруг юнита
\item распределение массива равномерно по всей карте
\end{mintemize}

В рамках дипломной работы был освоен вопрос гетерогенных вычислений,
написаны необходимые библиотеки-обёртки, разарботанно програмно-математическое
обеспечение для моделирования работы массива беспилотных 
летательных аппаратов. Произведены 9 экспериментов для 3х вараций алгоритма
построения карты для 3х количеств юнитов (16,33,50) со схожими моделями местности.
Произведён анализ результатов работы алгоритма по двум критериям:
\begin{mintemize}
\item соотношение изведанных секторов карты к общему количеству
\item средний интервал обновления известных секторов карты
\end{mintemize}
