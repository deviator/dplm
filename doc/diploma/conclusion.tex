\section{Заключение}

В задаче разработки алгоритмов работы МБПЛА существует множество аспектов.
Для большинства алгоритмов крайне сложно или невозможно доказать оптимальность.
Это связанно с большим количеством критериев и случайных факторов, а некоторые
из задач могут относиться к классу np-полных, для которых нельзя найти
решение за полиномиальное время. В этой ситуации нужно идти на компромисы и 
подбирать субоптимальные решения, основанные на неточных алгоритмах.
Без инструмента моделирования работы МБПЛА на компьютере такие решения
нельзя было бы проверить.

В данной работе было реализовано ПМО для моделирования работы МБПЛА и 3
алгоритма разведки местности.

~~~~~
* какие-то реализованны, какие-то нет
* самым сложным является их совмещение
* рассмотрены без математического доказательства
* проведены эксперименты на 2х типах алгоритмов (рандом и нет)
* тема сложная, для детального рассмотрения требуется больше
  времени
