\section{Массив БПЛА}

На данный момент массивы БПЛА не используются в том виде,
о котором идёт речь. В первую очередь, из-за технических
ограничений:
\begin{mintemize}
    \item отсутствие необходимых сенсоров
    \item отсутствие подходящей по соотношению быстродействие/масса
        вычислительной аппаратуры
    \item не проработанные алгоритмы
    \item большая стоимость единицы
\end{mintemize}

В этой работе делается попытка способствованию
устранению одного из ограничений использования МБПЛА:
алгоритмы работы массива.

\subsection{Имеющиеся технические решения}

???

\subsection{Видение будущих технических решений}

Конкретно обозначим задачи, решаемые МБПЛА:

\begin{mintemize}
\item Построение информационного поля участка местности
\item Поддержка актуальности, полноты и достоверности информационного поля
\item Обеспечение высокоскоростного канала радиосвязи
\item Обеспечение доступа к информационному полю личному составу
\end{mintemize}

Разберём эти задачи подробней.

\subsubsection{Построение информационного поля участка местности}

В понятие информационно поля участка местности могут входить:

\begin{mintemize}
\item Геометрическая информация об объектах (постройках, ландшафте)
\item Визуальная информация (изображения с камер и тепловых сенсоров юнитов)
\item Информация о радиоэфире на местности
\item Информация о премещении объектов
\end{mintemize}

Для получения информации о геометрии объектов на местности юниты могут
использовать различные технологические решения, отличающиеся точностью,
скоростью, ценой. Можно предположить, что реализацией подобных сенсоров на
борту юнита МБПЛА будет уже существующая, доработанная технология.

Рассмотрим варианты таких технологий:

\begin{mintemize}
\item Стереозрение

    плюсы:
    \begin{mintemize}
    \item дешевизна
    \item малая масса
    \end{mintemize}
    минусы:
    \begin{mintemize}
    \item высокая зависимость от погодных условий, времени суток
    \item возрастание ошибки с дальностью
    \item высокая алгоритмическая нагрузка на вычислитель
    \end{mintemize}
\item ЛИДАРы

    плюсы:
    \begin{mintemize}
    \item точность измерений
    \item независимость от времени суток
    \end{mintemize}
    минусы:
    \begin{mintemize}
    \item измерение дальности в одной угловой координате
    \item присутствие механических вращающихся частей
    \end{mintemize}

\item Времяпролетная камера (Time-Of-Flight)

    плюсы:
    \begin{mintemize}
    \item точность измерений
    \item дальность измерений в сравнении со стереозрением (до 1км)
    \end{mintemize}
    минусы:
    \begin{mintemize}
    \item на данный момент малая разрешающая способность ($640 \times 480$)
    \item измерения проводятся в конусе обзора
    \end{mintemize}
\end{mintemize}

Возможно, понадобится использовать комбинации из нескольких технологий.

Получение визуальной информации сейчас реализуемо с помощью камер. В этом
вопросе в будущем улучшения могут затронуть разрешающую способность, чувствительность,
динамический диапазон сенсора, характеристики оптики.

Информацию о радиоэфире можно получать с помощью радиоприёмников. Юниты, находящиеся
рядом, могут настраиваться на разные длины волн, для большего по диапазону
локального охвата эфира. Так же, используя алгоритмы триангуляции можно, 
с помощью нескольких юнитов, настроенных на одну волну, определять местоположение
источника радиосигнала.

Получение информации о перемещении объектов как в зоне видимости,
так и постфактум, является задачей больше алгоритмической, нежели технической.
Определение перемещения в зоне видимости можно выполнять рядом корреляционных
алгоритмов, адаптированных к трёхмерному описанию пространства карты.
Изменения неподвижной геометрии между обследованиями можно расценивать как
перемещения.

\subsubsection{Поддержка информационного поля}

После построения карты местности может возникнуть необходимость
поддержки актуальности, полноты и достоверности информационного поля,
наблюдения за местностью в течении заданного времени.
Условием такого наблюдения может быть постоянное обновление информации
об интересующих участках местности. Такая необходимость может возникнуть
в связи с возможными изменениями информационного поля:
\begin{mintemize}
\item изменение положения обследованных объектов 
\item изменение интенсивности и/или положения источников радиосигнала
\item постоянное слежение за подвижными объектами
\end{mintemize}

\subsubsection{Обеспечение связи и доступа к информационному полю}

Обеспечение высокоскоростного канала радиосвязи и обеспечение
доступа к информационному полю личному составу тесно связанны:
обеспечение доступа к информационному полю малоэффективно при
плохой связи внутри МБПЛА.

Решение этой задачи сводится к разрешению вопроса о позиционированнии юнитов
в пространстве и времени. Тоесть в каждый момент времени, для обеспечения 
наилучшей связи, необходимо раполагать каждый юнит в прямой видимости и 
на расстоянии не более определённого порога $L$ от нескольких соседних юнитов.
Это позволит использовать высокочастотную радиосвязь, которая позволяет
передавать больше информации в единицу времени.

Для обеспечения доступа личному составу к информационному полю можно принять
каждого бойца как юнит, починяющийся другим правилам перемещения. Ближайшие
юниты должны выстраиваться таким образом, чтобы радиоприёмники личного состава
были в прямой видимости и на расстоянии не более порога $L$.

В этом вопросе так же важна техническая сторона. Необходимы радиомодули,
позволяющие создавать mesh соединения, протоколы, контролирующие нагрузки
передачи данных и так далее.

\subsection{Аспекты решений поставленных задач} 

Так как решение задачи построения информационного поля
участка местности подразумевает хранение и обработку большого объёма
данных, появляется потребность в компактных и мощных вычислительных
устройствах, системах хранения, системах связи и т.д. 
Предполагается, что подобная аппаратура будет
стоять на каждом из юнитов.

\subsubsection{Разделение данных}

Каждый юнит должен хранить какую-то часть информационного поля.
При этом иформация должна дублироваться между несколькими юнитами в
зависимости от требования к сохранности информации в случае потерь
юнитов во время выполнения задачи. Юниты, среди которых дублируется
информация, должны быть разнесены пространственно в разные участки
для уменьшения вероятности потери информации. Если всё же информация
была утеряна, система может её восполнить с помощью других юнитов и 
распределить между оставшимися юнитами. Для доступа к информации юниты
должны быть связанны в единую сеть радиоканалом. У личного состава
должна быть возможность доступа к этой информации. 

На данный момент уже есть наработки по подобному разделению информации (NoSQL).

\subsubsection{Разделение вычислений}

При получении данных с датчиков юнит может как самостоятельно
её обработать, так и поручить обработку системе, которая
распределит задачу между несколькими юнитами, если они не
занимаются обработкой в данный момент.

\subsubsection{Представление карты}

Карта местности может содержать разнородную информацию:
\begin{mintemize}
    \item полигональные поверхности, построенные на основе анализа
        данных с сенсоров
    \item пространственная сетка с данными от тепловизоров
    \item пространственная сетка с данными о радиоэфире
    \item пространственная сетка с звуковой информацией
    \item подвижные объекты, как отдельные единицы карты
\end{mintemize}

Пространственная сетка это трёхмерная матрица значений, заполняемая
юнитами на основе собственного положения и информации от сенсоров.

\subsubsection{Локальная система позиционирования}

Каждый юнит должен знать свой фазовый вектор и других юнитов.
Но это в реальности не возможно, мы можем знать только
оценку этих параметров. Необходимо комплексировать данные с разных
юнитов для коррекции оценки одного юнита. Относительное положение
может быть вычесленно на основе интенсивности радиосвязи.

Так же коррекция и комплексирование может происходить на основе
изведанной местности, при условии, что она не подвижна.

\subsubsection{Построение маршрута обхода местности}

Решение задачи обхода местности по сути затрагивает все основные задачи
МБПЛА:

\begin{mintemize}
\item Оптимальный маршрут при начальной разведке местности
\item Расстановка и перемещение юнитов при поддержке
    актуальности информационного поля
\item Расстановка юнитов для обеспечения связи с учётом требований
\end{mintemize}

Работу МБПЛА можно разбить на 2 этапа:
\begin{mintemize}
\item Обследование местности (динамический)
\item Слежение и обеспечение связи (статический)
\end{mintemize}

На динамическом этапе добавляется очень много информации в систему.
В основном это информация о геометрии объектов. На статическом этапе
должны отслеживаться изменения карты, производиться поддержка связи.

Эти 2 этапа могут проходить параллельно или последовательно, в случае
нехватки юнитов. Для параллельного решения задачи необходимо разделить
массив на 2 группы, в зависимости от критериев, предъявляемых 
решению задачи: быстро обследовать или сразу занимать позиции слежения и 
обеспечения связи. Соответственно первая группа занимается реализацей
задач динамического этапа, а вторая группа реализацией задач
статического этапа. Некоторые условия, предъявляемые перемещению юнитов,
при реализации задач динамического этапа, могут не выполняться, например
условие прямой видимости, достаточного для обеспечения хорошей связи,
количества юнитов. Такое послабление позволит быстрее обследовать
местность. Но может возникнуть угроза внешнего деструктивного воздействия
на юниты (их могут сбивать). В таком случае нельзя допускать, чтобы 
какой либо юнит оставался без присмотра других. Это позволит, в случае
происшествия, лучше определить опасную зону, сразу пометив её на карте.
Также при таком параметре конфигурации МБПЛА динамический и статический
этап будут выполняться неразрывно друг от друга: слежение и связь будут
сразу организовываться над только что изведанной местностью.

\subsubsection{Поддержка высокоскоростной связи}

Солдатам может понадобится транслировать видео или другую
"тяжёлую" информацию командиру, находясь при этом в месте,
где прямая радиосвязь не обеспечивает надлежащего качества.
Для этого юниты выстраиваются в цепь от источника до приёмника
сигнала и пропускают через себя трафик. Юниты выстраиваются так,
что каждый из них имеет в прямой видимости не менее 2-х других
юнитов, являющихся элементами цепи связи.

\subsubsection{Взаимосвязь и иерархия элементов МБПЛА}

Решение задач реализации работы МБПЛА должно быть возложенно на элементы
самого МБПЛА. Юниты должны разбиваться на группы, напрмер
от 10 до 20 юнитов в группе, мастер-юниты групп должны также собираться
в группы. Мастер-юнит должен выполнять роль командира. Мастер-юнит для
группы других мастер-юнитов не должен быть командиром в более низкой по
иерархии группе. Это позволит распределить нагрузку, так как один мастер-юинт
управляет только одной группой. Группа не может состоять из юнитов разных
иерархических слоёв. В итоге мы получаем дерево иерархии, листьями которой
остаются юниты, выполняющие непосредственно задачи сбора информации.

Задачи, выполняемые коммандиром (мастер-юнитом).

\begin{mintemize}
\item Пространственное распределение юнитов
\item Распределение вычислительной нагрузки из системы
\item Распределение информации для хранения
\item Принятие решений на основе управляющих команд оператора
\end{mintemize}

Для обеспечения робастности МБПЛА каждый юнит, при необходимости, должен
иметь возможность стать мастер-юнитом. Это позволит в случае потери
мастер-юнита быстро его заменить. Выбор юнита на роль мастер-юнита в начале
работы массива может происходить по порядковому номеру. При потере
мастер-юнита по набору критериев:

\begin{mintemize}
\item тип выполняемой на данный момент задачи
\item геометрическое расположение юнита
\item величина вклада в выполняемую задачу
\end{mintemize}

Информация, над которой оперирует мастер-юнит, должна быть распределена по
МБПЛА. Это позволит в случае потери мастер-юнита сохранить информацию
и быстро его заместить другим юнитом, которому нужно будет только
прекратить выполнение текущей задачи и перейти к обработке задач управления.

Становиться мастер-юнитом могут только простые юниты, которые не заняты в
обеспечении работы группы. Это позволит снизить количество переключений.

Реализация поведения каждого из юнитов должна включать в себя
несколько режимов работы:

\begin{mintemize}
\item Автономный режим: запись собираемых данных на борт
\item Штатный режим: обеспечение связи, распределение собранных данных
\item Мастер режим: передача управляющих команд подопечной группе
\end{mintemize}

Работой всего массива управляет оператор. Операторов может быть несколько.
В случае нескольких операторов самый главныю юнит должен решать задачу
реализации команд каждого из операторов, если они не противоречат друг другу.
Если противоречий нет, возможен вариант совмещения выполнения команд, либо
полная раздельная реализация различными группами. Количество юнитов, необходимых
для выполнения той или иной команды выбирает главный юнит, исходя из самой 
команды, критериев выполнения, приоритета и тд.
