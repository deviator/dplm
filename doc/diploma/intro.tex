\section{Введение}

Всё чаще возникает проблема ведения боевых действий
в черте города. В связи с этим растёт необходимость в технических
решениях, которые позволят быстро и безопасно разведать местность.
Массив миниатюрных и дешёвых БПЛА может справиться с этой задачей 
лучше, как никто другой. Также такой массив может решать задачи недоступные
для решения с помощью классических беспилотников для наблюдения, а именно: 
\begin{mintemize}
    \item построение трёхмерной динамической карты местности
    \item исследование помещений
    \item непрерывный трекинг подвижных объектов
\end{mintemize}

Также к плюсам такой системы можно отнести высокую робастность в плане
устойчивости к потерям единиц, так как количество единиц в массиве
практически не ограниченно, а потеря даже 90\% массива не сможет
вывести систему из строя до конца. Качество решаемой задачи
будет плавно падать с падением количества единиц массива.

В понятие информационного поля местности также может входить
и возможность доступа оператора к данным датчиков любого из юнитов, например,
к камере, работающей в оптическом диапазоне.

\subsection{Постановка задачи}

В совокупности массив беспилотных летательных аппаратов является сложной,
многоуровневой системой. Задачи, решаемые МБПЛА являются составными.
Их решение не всегда можно чётко описать математически, так как они не
являются чётко формализованными.  В этой ситуации разработка алгоритмов
управления МБПЛА может происходить только при возможности имитационного
моделирования.

Отсутствие в настоящем времени реализаций подобного рода систем заставляет
выбирать параметры оснащения юнитов для имитационного моделирования, исходя
из логического и алгоритмического решения задач МБПЛА.

В целях упрощения имитационного моделирования будем считать, что:

\begin{mintemize}
\item каждый юнит имеет непосредственно информацию о своём положении и ориентации в пространстве
\item у каждого юнита есть информация о других юнитах массива
\item имеются все необходимые сенсоры и датчики (времяполётная камера, тепловизор и тд)
\end{mintemize}

Для имитационного моделирования необходимо:

\begin{mintemize}
\item Определить математическую модель юнита
\item Разработать ПМО для моделирования МБПЛА
\item Разбить основные задачи МБПЛА на подзадачи
\item Реализовать решения задач и подзадач в симуляторе
      в целях проверки
\end{mintemize}
