\subsection{Результаты моделирования}

В работе оценивалась работа 2х алгоритмов:
\begin{mintemize}
\item режим разведки (построение карты)
\item режим наблюдения (поддержание актуальности карты)
\end{mintemize}

Как критерий оценки использовалось состояние карты на момент
времени $T$ работы системы. Для режима построения карты это
заполненость карты в процентах, для режима наблюдения это
средний временной интервал между текущим временем и последним
временем проверки, а так же площадь покрытия наблюдения.

Графики строились в программе \verb|gnuplot|, информация для
построения генерировалась ПМО.

\newpage
\subsubsection{Режим построения карты}

\newpage
\subsubsection{Режим наблюдения}
