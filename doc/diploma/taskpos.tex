\section{Постановка задачи}

В совокупности массив беспилотных летательных аппаратов является сложной,
многоуровневой системой. Задачи, решаемые МБПЛА являются составными.
Их решение не всегда можно чётко описать математически, так как они не
являются чётко формализованными.  В этой ситуации разработка алгоритмов
управления МБПЛА может происходить только при возможности имитационного
моделирования.

Отсутствие в настоящем времени реализаций подобного рода систем заставляет
выбирать параметры оснащения юнитов для имитационного моделирования, исходя
из логического и алгоритмического решения задач МБПЛА.

В целях упрощения имитационного моделирования будем считать, что:

\begin{mintemize}
\item каждый юнит имеет непосредственно информацию о своём положении и ориентации в пространстве
\item у каждого юнита есть информация о других юнитах массива
\item имеются все необходимые сенсоры и датчики (времяполётная камера, тепловизор и тд)
\end{mintemize}

Для имитационного моделирования необходимо:

\begin{mintemize}
\item Определить математическую модель юнита
\item Разработать ПМО для моделирования МБПЛА
\item Разбить основные задачи МБПЛА на подзадачи
\item Реализовать решения задач и подзадач в симуляторе
      в целях проверки
\end{mintemize}

???
