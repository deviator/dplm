\section{Математическая модель юнитов}

За основу была взята концепция БПЛА с вертикальным взлётом (коптер или вертолёт),
то есть юнит имеет возможность зависать в точке и резко изменять направление движения.

Так как для реализации алгоритмов не требуется физически корректная симуляция
поведения юнитов представим их как материальные точки.

Дифференциальное уравнение движения:
$$
\left\{
    \begin{array}{l l}
    \dot{\vec{p}}  & = \vec V \\
    \dot{\vec{V}}  & = \vec a + \vec g
    \end{array}
\right.
$$

где:

$\vec p = (x,y,z)^T$ -- координаты юнита,

$\vec V = (V_x,V_y,V_z)^T$ -- скорость юнита,

$\vec a$ -- ускорение, вычисленное, как сумма всех сил делённая на массу,

$\vec g$ -- ускорение свободного падения,

$$ \vec a = \frac{1}{m} \cdot F_{ctrl} $$

где:

$m$ -- масса юнита,

$F_{ctrl}$ -- управляющее воздействие.

Вычисляется управляющее воздействие на основе работы алгоритмов управления.


