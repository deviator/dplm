\subsection{Математическая модель движения юнитов}

За основу была взята концепция БПЛА с вертикальным взлётом (коптер или вертолёт),
то есть юнит имеет возможность зависать в точке и резко изменять направление движения.

Так как для реализации алгоритмов не требуется физически корректная симуляция
поведения юнитов, представим их как материальные точки.

Дифференциальное уравнение движения:
$$
\left\{
    \begin{array}{l l}
    \dot{\vec{p}}  & = \vec V \\
    \dot{\vec{V}}  & = \vec a + \vec g
    \end{array}
\right.
$$

где:

$\vec p = (x,y,z)^T$ -- координаты юнита,

$\vec V = (V_x,V_y,V_z)^T$ -- скорость юнита,

$\vec g$ -- ускорение свободного падения,

$\vec a = \frac{1}{m} \cdot \sum_0^N \vec F_i$ -- ускорение,
вычисленное, как сумма всех сил делённая на массу,

где:

$m$ -- масса юнита,

$\vec F$ -- сила, действующая на юнит

В состав сил, действующих на юнит входят:

\begin{mintemize}
\item управляющее воздействие $\vec F_C$, вычисляется на основе работы алгоритмов
\item сила сопротивления воздуха $\vec F_D = \vec V C_{x0} \frac{\rho |V|}{2} S$
\end{mintemize}

Для юнита берётся обобщённый параметр $C_{x0}S$, подбираемый эмпирически.
Плотность воздуха в процессе симуляции не изменна и равна 1.
