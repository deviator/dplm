\subsection{Экономическая часть}

Затраты на разработку продукта включают в себя:

\begin{mintemize}
\item заработная плата
\item страховые взносы
\item плата за аренду помещения
\item амортизация оборудования
\item накладные расходы
\end{mintemize}

\subsubsection{Заработная плата}

Примем, что проект разрабатывается одним разработчиком средней
квалификации со средней часовой ставкой 400 руб.
Тогда рассчитаем затраты на заработную плату с учетом
продолжительности работ, приведенной в предыдущем разделе. Будем
учитывать, что разработка ведется в рамках 8-часового рабочего дня.

$$ C_\text{ЗП} = C_\text{ЧС} \cdot T $$

где:

$C_\text{ЧС}$ -- часовая ставка (руб),

$T$ -- общее время работ.

$$ T = 47 \cdot 8 = 376 \text{ч} $$
$$ C_\text{ЗП} = 400 \cdot 376 = 150400 \text{руб} $$

\subsubsection{Страховые взносы}

На данный момент размеры страховых взносов, уплачиваемых
организациями, устанавливаются федеральным законом \verb|№212-ФЗ| <<О
страховых взносах в Пенсионный фонд Российской Федерации, Фондсоциального страхования Российской Федерации, Федеральный фонд
обязательного медицинского страхования и территориальные фонды
обязательного медицинского страхования [редакция от 28.12.2010]>>.
Будем считать, что организация -- плательщик страховых взносов
попадает в пункт 4 части 1 данного федерального закона, а именно:
"...образовательными учреждениями высшего профессионального образования..."

\newpage

Таким образом, будут применяться пониженные тарифы

\newcommand{\tp}[2]%
{
    \begin{tabpage}{#2}
    #1
    \end{tabpage}
}

\vspace{1em}

\begin{tabular}{|c|c|}
\hline
Наименование & 2014 год \\
\hline
\tp{Пенсионный фонд Российской Федерации}{12cm} & 8.0 \% \\
\hline
\tp{Фонд социального страхования Российской Федерации}{12cm} & 2.0 \% \\
\hline
\tp{Федеральный фонд обязательного медицинского страхования}{12cm} & 2.0 \% \\
\hline
\tp{Территориальные фонды обязательного медицинского страхования}{12cm} & 2.0 \% \\
\hline
\end{tabular}

$$C_\text{СВ} = C_\text{ЗП} \cdot ( K_\text{ПФ} + K_\text{ФСС} +
K_\text{ФФОМС} + K_\text{ТФОМС} )$$

где:

$C_\text{ЗП}$ -- заработная плата,

$K_\text{ПФ}$ -- пенсионный фонд РФ,

$K_\text{ФСС}$ -- фонд социального страхования РФ,

$K_\text{ФФОМС}$ -- федеральный фонд обязательного медицинского страхования

$K_\text{ТФОМС}$ -- территориальные фонды обязательного медицинского страхования

$$C_\text{СВ} = 150400 \cdot ( 0.08 + 0.02 + 0.02 + 0.02 ) = 21056 \text{руб}$$

\subsubsection{Плата за аренду помещения}

Рассчитаем стоимость аренды по следующей формуле:
$$ C_\text{АП} = S \cdot C_\text{А} \cdot T $$
где:

$S$ -- площадь помещения,

$C_\text{А}$ -- ставка аренды $\left(\frac{\text{руб}}{\text{мес}\cdot\text{м}^2}\right)$,

$T$ -- срок аренды (мес).

Площадь помещения, в котором производятся работы, равна $27.95 \text{м}^2$.

Ставка аренды составляет $800 \frac{\text{руб}}{\text{мес}\cdot\text{м}^2}$.

Срок аренды составляет 2 мес.

Таким образом:

$$ C_\text{АП} = 27.95 \cdot 800 \cdot 2 = 44720\text{руб} $$

\subsubsection{Амортизация оборудования}

Амортизационные отчисления осуществляются ежемесячно с целью
возмещения износа оборудования.

Выберем линейный способ амортизационных отчислений.

Оборудование, используемое при выполнении работы, представлено в таблице.

\vspace{1em}

\begin{tabular}{|c|c|c|c|}
\hline
Наименование & Количество & Ресурс, мес. & Цена, руб. \\
\hline
ПК & 1 & 60 & 60 000 \\
\hline
Монитор & 1 & 60 & 15 000 \\
\hline
Клавиатура & 1 & 36 & 2 000 \\
\hline
Мышь & 1 & 36 & 1 500 \\
\hline
Кондиционер & 1 & 120 & 70800 \\
\hline
\end{tabular}

\vspace{1em}

Расчёт амортизационных отчислений производится по формуле:

$$ C_\text{АО} = \sum_{i=0}^{N} \frac{C_{\text{О}i} \cdot T_i}{T_{\text{Р}i}} $$

где:

$C_\text{О}$ -- стоимость оборудования (руб),

$T$ -- время использования оборудования (мес),

$T_\text{Р}$ -- ресурс оборудования (мес).

Подставим значения:

$$ C_\text{АО} = \frac{60000 \cdot 2}{60} + \frac{15000 \cdot 2}{60}
+ \frac{2000 \cdot 2}{36} + \frac{1500 \cdot 2}{36} +
\frac{70800 \cdot 2}{120} = 3874.5 \text{руб} $$

\subsubsection{Накладные расходы}

При расчете общей себестоимости необходимо учесть такие расходы,
как затраты на электроэнергию, потребляемую техникой, затраты на
отопление помещения и необходимые материалы.

\vspace{1em}

\begin{tabular}{|c|c|c|}
\hline
Устройство & Количество & Потребляемая мощность, Вт \\
\hline
ПК & 1 & 1000 \\
\hline
Монитор & 1 & 35 \\
\hline
Лампа & 15 & 12 \\
\hline
Кондиционер & 1 & 1500 \\
\hline
\end{tabular}

\vspace{1em}

Стоимость $1\text{кВт} \cdot \text{ч}$ равна $4.5\text{руб}$.

Общее время работ равно $376\text{ч}$.

$$ C_\text{ЭЭ} = C \cdot T \cdot \sum_{i=0}^N K_i \cdot M_i $$

где:

$N$ -- количество приборов,

$K$ -- количество однотиптых приборов,

$M$ -- мощность одного прибора,

$C$ -- стоимость кВт$\cdot$ч,

$T$ -- общее время работы оборудования (ч).

Подставим значения в формулу:

$$ C_\text{ЭЭ} = 4.5 \cdot 376 \cdot (1 + 0.035 + 12 * 0.015 + 1.5) = 4593.78 \text{руб} $$

\vspace{1em}

Расчитаем затраты на отопление.

$$ C_\text{ОП} = S \cdot T \cdot C_\text{О} $$

где: 

$S$ -- площадь помещения,

$T$ -- период (мес),

$C_\text{О}$ -- тариф на тепловую энергию (руб/кв.м. в мес).

$$ C_\text{ОП} = 27.95 \cdot 2 \cdot 25 = 1397.5 \text{руб} $$

\vspace{1em}

Материалы используемые при разработке: 
\begin{mintemize}
\item упаковка бумаги -- 200 руб
\item другая канцелярия -- 100 руб
\end{mintemize}

\vspace{1em}

Суммарные накладные расходы:

$$ C_\text{НР} = C_\text{ЭЭ} + C_\text{ОП} + C_\text{М}
= 4593.78 + 1397.5 + 300 = 6291.28 \text{руб} $$ 

\subsubsection{Суммарная себестоимость}

Суммируя полученные значения затрат на заработную плату, страховые
взносы, арендную плату, амортизационные отчисления и накладные расходы,
получим общую себестоимость:

$$ C = C_\text{ЗП} + C_\text{СВ} + C_\text{АП} + C_\text{АО} + C_\text{НР} +
150400 + 21056 + 44720 + 3874.5 + 6291.28 = 226342 \text{руб} $$ 
