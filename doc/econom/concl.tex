\subsection{Заключение}

В разделе были рассмотрены планирование работ, а также оценка
себестоимости разрабатываемого программно-математического обеспечения.
Как можно наблюдать по сетевому графику, продолжительность работ
не выходит за рамки периода выполнения дипломной работы, и все работы
могут быть выполнены одним разработчиком в установленный срок.

Для наглядного представления приведем круговую диаграмму,
отображающую различные виды затрат.

\vspace{3em}

% зп = 0.6644 -- 234.184
% страховые взносы = 0.093 -- 33.48
% арендная плата = 0.198 -- 71.28
% амортизация = 0.0171 --  6.1596
% накладные расходы = 0.0278 -- 10.08

{
\def\zppart{234.184}
\def\svpart{33.48}
\def\appart{71.28}
\def\aopart{6.1596}
\def\r{3}
\def\offset{4}
\def\rs{0.3}

\begin{tikzpicture}
\filldraw[fill=red!30] (0,0) -- +(0:\r) arc (0:\zppart:\r) -- (0,0);
\filldraw[fill=green!30] (0,0) -- +(\zppart:\r) arc (\zppart:{\zppart+\svpart}:\r) -- (0,0);
\filldraw[fill=blue!30] (0,0) -- +({\zppart+\svpart}:\r) arc ({\zppart+\svpart}:{\zppart+\svpart+\appart}:\r) -- (0,0);
\filldraw[fill=magenta!30!yellow] (0,0) -- +({\zppart+\svpart+\appart}:\r) arc ({\zppart+\svpart+\appart}:{\zppart+\svpart+\appart+\aopart}:\r) -- (0,0);
\filldraw[fill=cyan!30] (0,0) -- +({\zppart+\svpart+\appart+\aopart}:\r) arc ({\zppart+\svpart+\appart+\aopart}:360:\r) -- (0,0);

\filldraw[fill=red!30] (\offset,{\r*0.8}) +(0,0) rectangle +(\rs,\rs);
\node[right] at ({\offset+\rs},{\r*0.8+\rs*0.5}) {$\text{заработная плата}$};

\filldraw[fill=green!30] (\offset,{\r*0.6}) +(0,0) rectangle +(\rs,\rs);
\node[right] at ({\offset+\rs},{\r*0.6+\rs*0.5}) {$\text{страховые взносы}$};

\filldraw[fill=blue!30] (\offset,{\r*0.4}) +(0,0) rectangle +(\rs,\rs);
\node[right] at ({\offset+\rs},{\r*0.4+\rs*0.5}) {$\text{арендная плата}$};

\filldraw[fill=magenta!30!yellow] (\offset,{\r*0.2}) +(0,0) rectangle +(\rs,\rs);
\node[right] at ({\offset+\rs},{\r*0.2+\rs*0.5}) {$\text{амортизация оборудования}$};

\filldraw[fill=cyan!30] (\offset,{\r*0.0}) +(0,0) rectangle +(\rs,\rs);
\node[right] at ({\offset+\rs},{\r*0.0+\rs*0.5}) {$\text{накладные расходы}$};
\end{tikzpicture}
}

\vspace{2em}

Как можно наблюдать, основная часть расходов связана с заработной и
арендной платами. Для того чтобы сократить эти расходы, можно нанять
сотрудника с более низкой почасовой ставкой, арендовать помещение
меньшей площади.

Так как рассматриваемая работа является научно-исследовательской,
она не предполагает прямой прибыли. Поэтому этап ценообразования не был
рассмотрен, исследование рынка не проводилось.
