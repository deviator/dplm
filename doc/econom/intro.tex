\subsection{Введение}

В данной дипломной работе проводится разработка алгоритма управления
массивом беспилотных летательных аппаратов (МБПЛА) и написание
симулятора для отладки алгоритма.

На данный момент разработка как такового МБПЛА является не реализуемой
задачей, в первую очередь из-за технических ограничений современного
оборудования.

В перспективе массивы беспилотных летательных аппаратов могут стать
незаменимым инструментом ведения боевых действий в черте города.
Так же данная технология может применяться и гражданскими. Основная
задача МБПЛА состоит в построении информационного поля участка
урбанистической местности. Информация, поступаемая от юнитов, имеет
минимальную временную задержку. В связи с этим, подразделение,
использующее МБПЛА, имеет возможность незамедлительно реагировать
на события. Покрытие местности зависит только от количества юнитов.
Гражданское применение МБПЛА видится при устранении последствий
природных и техногенных катастроф, когда необходимо постоянное
наблюдение изменений на участке местности.

В текущем разделе будут рассмотрены вопросы планирования разработки
и затрат на разработку программно-математического обеспечения.
