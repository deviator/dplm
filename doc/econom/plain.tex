\subsection{Организационная часть}

{
\begin{wrapfigure}{r}{0.25\linewidth}
\begin{flushright}
\begin{tikzpicture}[scale=0.65,->,>=stealth',shorten >=0pt,auto,node distance=2.5cm,
  thick,every node/.style={scale=0.65},
  mnode/.style={circle,minimum size=1cm,fill=blue!20,draw,font=\sffamily\bfseries},
  hnode/.style={circle,minimum size=0.5cm,fill=white,draw},
  mainp/.style={line width=2pt}]

\node[mnode,fill=white] (0) {start};
\node[mnode] (1) [below of=0] {1}; % постановка задачи 2
\node[mnode] (2) [below of=1] {2}; % построение архитектуры 3
\node[mnode] (3) [below of=2] {3}; % распределение работы 1
\node[mnode] (4) [below left of=1] {4}; % разработка ММ МБПЛА 1
\node[mnode] (5) [below of=4] {5}; % перечень обарудование 1
\node[mnode] (6) [below of=3] {6}; % графическая часть 7
\node[hnode] (p7) [below left of=3] {};
\node[mnode] (7) [below of=p7] {7}; % мат часть 4
\node[hnode] (p8) [below right of=7] {};
\node[mnode] (8) [below of=p8] {8}; % интеграция 2
\node[mnode] (9) [below of=8] {9}; % тест и дебаг 4
\node[mnode] (10) [below right=1.5cm and 1.5cm of 1] {10}; % постр алг упр 8
\node[hnode] (p11) [below of=9] {};
\node[mnode] (11) [below of=p11] {11}; % реализ в симуляторе 2
\node[mnode] (12) [below of=11] {12}; % тест и дебаг в сборе 4
\node[mnode] (13) [below of=12] {13}; % эксперименты 2
\node[mnode] (14) [below right of=6] {14}; % док граф 2
\node[mnode] (16) [below right=1.5cm and 1.5cm of p11] {16}; % док алг 2
\node[mnode,fill=white] (final) [below of=13] {final}; % док алг 2
\node[mnode] (15) [above left of=final] {15}; % док мат 2

\path[every node/.style={right,font=\sffamily\small}]
(0) edge[mainp] node {2} (1)
(1) edge[mainp] node {3} (2)
(2) edge[mainp] node {1} (3)
(1) edge node {1} (4)
(4) edge node {1} (5)
(3) edge[mainp] node {7} (6)
(3) edge (p7)
(5) edge (p7)
(p7) edge node {4} (7)
(7) edge (p8)
(6) edge[mainp] (p8)
(p8) edge[mainp] node {2} (8)
(8) edge[mainp] node {4} (9)
(1) edge node {8} (10)
(9) edge[mainp] (p11)
(p11) edge[mainp] node {2} (11)
(11) edge[mainp] node {4} (12)
(12) edge[mainp] node {2} (13)
(p11) edge node[right=0.2cm] {2} (16)
(6) edge node {2} (14)
(7) edge node {2} (15)
(13) edge[mainp] (final)
(15) edge (final)
;

\draw (10) -- ($(p11)+(3,3)$) -> (p11)
(14) -- ($(final)+(1.75,1.75)$) -> (final)
(16) -- ($(final)+(2.85,2.85)$) -> (final);

\end{tikzpicture}
\end{flushright}
\end{wrapfigure}

Организация работы над проектом состоит из описания необходимых работ
и распределии ресурсов на их выполнение. Для оптимизации процесса
построим сетевой график выполенния работ.

Ниже представлены работы, необходимые для выполнения проекта.
Справа сетевой график работ.

\vspace{1em}

\newcounter{taskNoCounter}
\def\taskNo{\addtocounter{taskNoCounter}{1}\arabic{taskNoCounter}}

\newcommand{\ttelem}[2]%
{
    \hline
    \taskNo & \begin{tabpage}{8.5cm}
    #1
    \end{tabpage} &#2 \\
}

\begin{flushleft}
\begin{tabular}{|c|c|c|}

\hline
№ & Работа по проекту & ч.д. \\

\ttelem{Постановка задачи, конкретизация критериев работы алгоритма}{2}

\ttelem{Построение архитектуры симулятора}{3}

\ttelem{Распределение работы по написанию симулятора}{1}

\ttelem{Разработка математической модели МБПЛА}{1}

\ttelem{Проработка перечня и параметров необходимого
для юнита оборудования, которое будет имулироваться}{1}

\ttelem{Написание графической части симулятора}{7}

\ttelem{Написание математической части симулятора}{4}

\ttelem{Интеграция частей симулятора}{2}

\ttelem{Тестирование и отладка симулятора}{4}

\ttelem{Построение алгоритма управления \lb МБПЛА}{8}

\ttelem{Реализация алгоритма управления МБПЛА в симуляторе}{2}

\ttelem{Тестирование и отладка системы в сборе}{4}

\ttelem{Экспериметы и анализ результатов}{2}

\ttelem{Написание документации по графической части симулятора}{2}

\ttelem{Написание документации по математической части симулятора}{2}

\ttelem{Написание документации по алгоритму управления}{2}

\hline
\end{tabular}
\end{flushleft}
}

\newpage

Критический путь выделен толсными стрелками
и составляет 27 человеко-дней. Общая трудоёмкость составляет 47 человеко-дней.
Принимая во внимание структуру сетевого графика, можно отметить,
что для реализации проекта необходимо 3 инженера-программиста.

