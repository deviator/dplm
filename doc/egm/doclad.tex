%@def.tex
\documentclass[a4paper,12pt,oneside]{article}
\usepackage[warn]{mathtext}
\usepackage[T2A]{fontenc}
\usepackage[utf8]{inputenc}
\usepackage[english, russian]{babel}
\usepackage{indentfirst}
\usepackage{amsmath, amsfonts, amssymb}
\usepackage{geometry}
\usepackage{hyperref}
\geometry{top=2cm}
\geometry{left=3cm}
\geometry{right=2cm}
\geometry{bottom=2cm}
\usepackage{graphicx}
\usepackage{epstopdf}
\usepackage{tikz}
\usetikzlibrary{arrows}
\usetikzlibrary{calc}

\renewcommand{\vec}{\overline}
\renewcommand{\phi}{\varphi}

\newenvironment{mintemize}%
{
    \vspace{-0.5em}
    \begin{itemize}
        \setlength\itemsep{-0.2em}
}
{
    \end{itemize}
    \vspace{-0.5em}
}

\graphicspath{{pics/}}


\newenvironment{docsec}[1]{%
\section{#1}
\begin{mintemize}
}
{ \end{mintemize} }

\begin{document}

\begin{docsec}{Что такое массив БПЛА}
\item МБПЛА -- группа однотипных, сверхлёгкий, дешёвых
\item Юнит -- элемент МБПЛА
\item Количество юнитов может исчисляться сотнями и тысячами
\item Большая робастность системы
\end{docsec}

\begin{docsec}{Для чего нужен МБПЛА}

\item боевые действия в черте города
\item ограничения классических БПЛА

\end{mintemize}

\begin{mintemize}

\item построение трёхмерной динамической карты местности
\item исследование помещений
\item непрерывный трекинг подвижных объектов
\item патруль территории
\item построение дополненной реальности \newline ("взор сквозь стены")

\end{docsec}

\begin{docsec}{Реализация МБПЛА}

\item на данный момент подобных решений нет из-за отсутствия
    \begin{mintemize}
        \item сенсоров
        \item лёгкой, энергоэффективной, быстродействующей вычислительной
              аппаратуры
        \item алгоритмов
        \item технологии изготовления в поле
    \end{mintemize}

\item
    \begin{mintemize}
        \item Построение информационного поля участка местности
        \item Поддержка актуальности, полноты и достоверности информационного поля
        \item Обеспечение высокоскоростного канала радиосвязи
    \end{mintemize}

\end{docsec}

\begin{docsec}{Аспекты реализации}

\item для реализации этих задач нужно решить такие вопросы как
    \begin{mintemize}
        \item Разделение данных
        \item Разделение вычислений
        \item Локальная система позиционирования
        \item Построение маршрута обхода местности
        \item Задача позиционирования в пределах видимости
        \item Взаимосвязь и иерархия элементов МБПЛА
    \end{mintemize}

\end{docsec}

\begin{docsec}{В работе рассматривается}

\item Ставится задача построить алгоритм обследования
      местности (построения карты) и поддержания актуальности
      карты, разработать ПМО для моделирования и провести
      эксперименты

\item Формализация карты местности, какие поля необходимы
\item Модель юнита, его оснащение, какие необходимы элементы
\item Алгоритм обхода выпуклых препядствий
\item Алгоритм построения карты местности
    с вариациями
\item Реализация ПМО и проведение экспериментов

\end{docsec}

\begin{docsec}{Модель карты местности}

    \item Предполагается что участок местности ограничен по ширине,
    глубине и высоте.
    Он разбивается на прямоугольные сектора, каждый из которых хранит
    информацию об единице объёма:

    \begin{mintemize}
    \item сколько раз был проверен сектор
    \item когда была последняя проверка
    \item результат проверки (занят ли сектор)
    \end{mintemize}

    \item Реализация карты представляет собой 1-мерный массив,
    к которому можно обратиться с помощью трёх индексов-координат.
    Так же для карты имеется матрица трансформации координат
    из локальных (индексов) в глобальные (метры).

\end{docsec}

\begin{docsec}{Модель юнита}
    \item За основу была взята концепция БПЛА с вертикальным взлётом
    \item У каждого юнита на борту имеется датчик измеряющий "карту глубин"
    \item Каждый юнит знает о положении остальных юнитов системы
    \item В список действующих на юнит сил входят:
        \begin{itemize}
            \item Сила сопротивления воздуха
            \item Управляющее воздействие
            \item Коррекция по ближайшим опасным точкам
        \end{itemize}
\end{docsec}

\begin{docsec}{Заполенние карты}
        \item 1. получаем карту глубин
        \item 2. с помощью обратной матрицы перспективной трансформации
            переводим каждую точку на карте глубин в точку в пространстве
        \item 3. для каждой точки строим отрезок из камеры в эту точку
        \item 4. проходим по отрезку с небольшим шагом
            \begin{itemize}
            \item на каждом шаге приводим точку к системе координат карты
            \item если сектор не исследован помечаем сектор карты как пустой
            \end{itemize}
        \item 5. проверям длину отрезка
            \begin{itemize}
            \item если близко к предельному для сенсора значению по дальности
                то помечаем сектор на конце отрезка как пустой
            \item иначе помечаем сектор как заполненный
            \end{itemize}
        \item 6. инкрементируем количество проверок сектора и
            устанавливаем текущее значение времени как время последней проверки

        \item получается некий алгоритм 3х-мерной растеризации
\end{docsec}

\begin{docsec}{Коррекция по ближайшим опасным точкам}

\item такой алгоритм производится для всех точек ближе безопастного
    расстояния

\end{docsec}

\begin{docsec}{Управляющее воздействие}

\item Был выбран подход к построению перемещения юнитов по целевым точкам

\item управляющее воздействие вычисляется с помощью PID
\item на вход PID подаётся разница между текущим положением и целевой
      точкой
\item Вариации выбора целевой точки
    \begin{mintemize}
        \item Распределение по секторам карты групп юнитов \newline
        изаначально необходимо знать минимальное количество юнитов в системе,
        каждый юнит должен иметь порядковый номер, по которому ему присваивается
        регион, внутри региона юниты подгруппы делят между собой ещё один уровень
        разбиется. На этом уровне нет жёсткой фиксации, целевая точка для
        конкретного юнита выбирается как элемент подрегиона с шагом в
        количество юнитов в текущем регионе

        \item Поиск ближайших неизведанных регионов \newline
        берётся случайный угол и с шагом в 90 градусов на фиксированном
        расстоянии от юнита выбираются 4 региона, в которых подсчитывается 
        количество неизвестных точек, вычисляется их центр, выигрывает
        тот регион в котором больше неизвестных точек, его центр берётся
        как новая целевая точка для юнита

        \item Случайный выбор точки в пределах карты
    \end{mintemize}

\item момент выбора новой целевой точки определяется по дисперсии
      изменения координат, если юнит мало перемещается,
      значит он либо нашёл угол (невыпукую фигуру), который не может
      обойти, либо достиг целевой точки. Когда дисперсия перестаёт
      превышать порог, происходит выбор новой точки.

\end{docsec}

\end{document}
