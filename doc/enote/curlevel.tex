\newpage
\section{Массив БПЛА}

На данный момент массивы БПЛА не используются в том виде,
о котором идёт речь. В первую очередь из-за технических
ограничений: 
\begin{mintemize}
    \item отсутствие необходимый сенсоров
    \item отсутствие подходящей по соотношению быстродействие/масса
        вычислительной аппаратуры
    \item не проработанные алгоритмы
    \item большая стоимость еденицы
\end{mintemize}

В этой работе делается попытка способствованию
устранию одного из этих ограничений (проработка алгоритмов).

\subsection{Видение будующих технических решений}

Так как решение задачи построения информационного поля
участка местности подразумевает хранение и обработку большого объёма
данных, появляется потребность в компактных и мощьных вычислительных
устройствах. Предполагается, что подобные бортовые вычислители будут
стоять на каждом из юнитов. Хранение информации также будет
разделено по юнитам с некоторым дублированием. На данный момент уже
есть наработки по подобному разделению информации (NoSQL).

\subsection{Разделение данных}

Каждый юнит должен хранить какую-то часть информационного поля.
При этом иформация должна дублироваться между несколькими юнитами в
зависимости от требования к сохнанности информации в случае потерь
юнитов во время выполнения задачи. Юниты, среди которых дублируется
информация, должны быть разнесены пространственно в разные участки,
для уменьшения вероятности потери информации. Если всё же информации
была утеряна, система может её восполнить с помощью других юнитов и 
распределить между другими юнитами. Для доступа к информации юниты
должны быть связанны в единую сеть радиоканалом. У личного состава
должна быть возможность доступа к этой информации. 

\subsection{Разделение вычислений}

При получении данных с датчиков юнит может как самостоятельно
её обработать, так и поручить обработку системе, которая
распределит задачу между несколькими юнитами, если они не
занимаются обработкой в данный момент.

\subsection{Карта местности}

Карта местности может содержать разнородную информацию:
\begin{mintemize}
    \item полигональные поверхности, построенные на основе анализа
        данных с дальномеров
    \item пространственная сетка с данными от тепловизоров
    \item подвижные объекты, как отдельные еденицы
\end{mintemize}

\subsection{Локальная система позиционирования}

Каждый юнит должен знать фазовый вектор свой и других юнитов.
Но это в реальности это не возможно, мы можем знать только
оценку этих параметров. Необходимо комплексировать данные с разных
юнитов для коррекции оценки одного юнита.

\subsection{Построение маршрута обхода местности}

В данной работе как раз и рассматриваются алгоритмы построения
карты и маршрутов перемещения юнитов по карте, с учётом того,
что не вся карта известна.
