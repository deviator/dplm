%@def.tex
\documentclass[a4paper,12pt,oneside]{article}
\usepackage[warn]{mathtext}
\usepackage[T2A]{fontenc}
\usepackage[utf8]{inputenc}
\usepackage[english, russian]{babel}
\usepackage{indentfirst}
\usepackage{amsmath, amsfonts, amssymb}
\usepackage{geometry}
\usepackage{hyperref}
\geometry{top=2cm}
\geometry{left=3cm}
\geometry{right=2cm}
\geometry{bottom=2cm}
\usepackage{graphicx}
\usepackage{epstopdf}
\usepackage{tikz}
\usepackage{tikz-3dplot}
\usepackage{color}

\usetikzlibrary{arrows}
\usetikzlibrary{calc}

\renewcommand{\vec}{\overline}
\renewcommand{\phi}{\varphi}

% список с корректированным межстрочным интервалом
\newenvironment{mintemize}%
{
    \vspace{-0.5em}
    \begin{itemize}
        \setlength\itemsep{-0.2em}
}
{
    \end{itemize}
    \vspace{-0.5em}
}

\graphicspath{{pics/}}

% назначение межстрочного интервала в verbatim
\makeatletter
\newcommand{\nextverbatimspread}[1]{%
    \def\verbatim@font{%
        \linespread{#1}\normalfont\ttfamily% Updated definition
            \gdef\verbatim@font{\normalfont\ttfamily}}% Revert to old definition
}
\makeatother

% ссылки на новой странице с 1
\usepackage{perpage}
\MakePerPage{footnote}
