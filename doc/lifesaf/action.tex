\subsection{Мероприятия по обеспечению условий труда}

\subsubsection{Организация рабочего места программиста}

Исходя из задач, стоящих перед инженером-программистом, расположение 
технических средств на рабочем месте может различаться. Так как основная
работа ведётся за компьютером исключим наличие приборов, которые не
относятся к комплектации ПК.

Определим рабочее пространство.

\begin{center}
\begin{tikzpicture}

\filldraw[fill=black!25!white] (4,2.5) +(0.8,-1.6) rectangle +(-0.8,2);

\filldraw[fill=gray,fill opacity=0.2] (0,0.4) +(-20:2.4) -- +(-20:6) arc (-20:200:6 and 4.5) -- +(20:3.6);

\filldraw[fill=gray,fill opacity=0.2] (0,0) -- +(20:4) arc (20:160:4) -- +(-20:4);

\filldraw[fill=gray,fill opacity=0.2] (0,0) -- +(50:3.5) arc (50:130:3.5) -- +(-50:3.5);

\filldraw[fill=black!25!white] (0,3.55) +(2.5,0) -- +(2.5,0.2) -- +(1.8,0.2) -- +(1.5,0.5)
                    -- +(-1.5,0.5) -- +(-1.8,0.2) -- +(-2.5,0.2) -- +(-2.5,0) -- +(2.5,0);

\filldraw[fill=black!25!white,rounded corners=0.4cm] (0,-0.3) +(-2.4,0.4) rectangle +(2.4,-0.4);
\filldraw[fill=black!25!white] (0,-0.3) +(0,0) circle (0.7cm);

\filldraw[fill=black!25!white] (0,2.2) +(-1.8,0) rectangle +(1.8,1);
\draw[step=1cm,gray,dashed,very thin] (-6.2,-2.5) grid (6.2,5.2);

\node at (0,2.7) {$\text{клавиатура}$};
\node at (0,3.7) {$\text{монитор}$};
\node at (4,2.5) {$\text{ПК}$};
\node at (0,1.3) {$\text{Зона 1}$};
\node at (-2.3,1.5) {$\text{Зона 2}$};
\node at (-4,0.5) {$\text{Зона 3}$};

\foreach \y in {-2,-1,0,1,2,3,4,5}
    \node[left] at (-6.1,\y) {%
    \pgfmathparse{10*\y}    % Evaluate the expression
    \pgfmathprintnumber[    % Print the result
        fixed,
        fixed zerofill,
        precision=0
    ]{\pgfmathresult}%
    };
\foreach \x in {-5,-4,-3,-2,-1,0,1,2,3,4,5}
    \node[below] at (\x,-2) {%
    \pgfmathparse{10*\x}    % Evaluate the expression
    \pgfmathprintnumber[    % Print the result
        fixed,
        fixed zerofill,
        precision=0
    ]{\pgfmathresult}%
    };

\end{tikzpicture}
\end{center}

\begin{mintemize}
\item Зона 1 -- Оптимальная зона моторного поля, то есть
    зона для расположения очень часто используемых и наиболее
    важных для работающего человека органов управления
\item Зона 2 -- Зона легкой досягаемости моторного поля,
    в которой размещаются часто используемые органы управления
\item Зона 3 -- Зона досягаемости моторного поля,
    в которой размещаются редко используемые органы управления.
\end{mintemize}

Клавиатура и монитор персонального компьютера располагаются в
оптимальной зоне так, чтобы обеспечить наилучшие условия труда,
потому что основное время работы над алгоритмом будет представлять
из себя работу за компрьютером.

Системный блок персонального компьютера можно расположить в зоне
досягаемости моторного поля, под столом, так как доступ к нему необходим
только при его включении и выключении.

Целесообразно монитор расположить в вертикальной плоскости под углом
$\pm 10^\circ$ от нормальной линии взгляда. Для меньшей утомляемости
зрения экран монитора должен быть матовым. Клавиатура компьютера
должна иметь угол наклона к горизонтали равный $12-15^\circ$.

Для обеспечения данных требований монитор должен быть установлен на
подставке, которая регулирует его наклон по вертикали.  Клавиатуру
следует использовать с предусмотренными конструкцией подставками,
которые позволяют устанавливать наклон клавиатуры в оптимальное значение.
Фактически на рабочем месте располагается TFT E-IPS монитор
NEC MultiSync LCD2490WUXi$^2$ 24.1', который регулируется и по
вертикали и по горизонтали, и клавиатура A4-tech, которая конструктивно
обеспечивают угол наклона в $\approx 12^\circ$ без подставки.

Пространственная организация рабочего места в основном определяется
размерами и формой сенсорного и моторного пространства, формой и
параметрами элементов рабочего места. Размеры и форма информационного
и моторного поля регламентированы \verb|СанПиН 2.2.2/2.4.1340-03|.
Очень важным фактором при организации рабочего места является выбор и
расположение рабочего кресла и рабочей поверхности, которые оказывают
непосредственное воздействие на производительность труда программиста.

В зависимости от роста меняются требования к высоте рабочего кресла и
рабочей поверхности. Оптимальной считается регулируемая высота рабочей
поверхности в пределах 670-800мм, при отсутствии регулировки -- 725мм.
Ширина рабочей поверхности должна быть не меньше, чем глубина рабочей плоскости,
и должна быть не менее 700мм. Размеры рабочей поверхности выбирается исходя из
габаритов и комплектации приборов, которые необходимо на ней расположить,
оптимальным при выполнении работ считается размер рабочей поверхности стола в
1600 на 800 мм. При длительно работе программиста за столом возможно ухудшение
кровоснабжения таза и ног, для этого необходимо предусмотреть специальную
подставку под ноги. Кроме того под столешницей рабочего стола предусматривается
свободное пространство для ног размерами: 600 на 500 на 650мм. Фактически высота
рабочей поверхности составляет 700мм, ширина рабочей поверхности составляет 700мм,
а длина – 1600мм, что соответствует нормам.

Тип сидения выбирают в виде рабочего кресла или стула, которые снабжены
подъемно-поворотными устройствами, обеспечивающими регулировку высоты сидения,
спинки, а также изменение угла наклона спинки. Рабочее кресло должно иметь
подлокотники. Используемые кресла имеют регулировки высоты сиденья, наклона спинки,
высоты подлокотников, высоты и наклона подголовника. Это позволяет обеспечить
оптимальное положение тела работающего человека.

\subsubsection{Микроклимат помещения}

В помещении установлен настенный кондиционер Mitsubishi Heavy SRK50ZMX-S.
Он имеет режимы охлаждения и нагрева, следовательно, в любое время года
в помещении поддерживается оптимальная температура, а функция 3D-Auto позволяет
сократить разброс температуры в помещении как по горизонтали, так и по вертикали,
за счёт трёхмерного управления воздушным потоком. Так же, каждую неделю
происходит уборка помещения от пыли и грязи.

\subsubsection{Шум}

В целях уменьшения количества шума, издаваемого системными блоками ПК, на нижнюю
поверхность опор системных блоков наклеены мягкие, цилиндрические пластины. Это
не значительно уменьшает устойчивость системного блока, как и не значительно уменьшает
уровень шума.

\subsubsection{Электроопасность}

Современные ПК обладают высокой степенью защиты от поражения электрическим током.
В шнуры питания системного блока и монитора входит провод заземление,
который потом соединяется с заземлением в розетке. В помещении все разетки заземлены.

\subsubsection{Пожароопасность}

Для обеспечения своевременной ликвидации локальных возгораний в помещении,
в специальном шкафу, располагается огнетушитель ОУ-2. На случай возникновения
пожара в помещении вывешена план-схема эвакуации людей из здания. Так же в 
помещении имеется пожарная сигнализация.

\subsubsection{Освещённость}

Помимо главного освещения на рабочих местах есть лампы, которые можно направлять
на поверхность, где требуется больше освещения.
