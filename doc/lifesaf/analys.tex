\subsection{Анализ условий работы}

\subsubsection{Характеристики помещения}

Разработка алгоритмов управления массивом беспилотных летательных аппаратов (МБПЛА)
и программного обеспечения для моделирования поведения МБПЛА проводится в помещении
размером $4.3 \cdot 6.5 = 27.95 \text{м}^2$, высота потолков составляет $3.2 \text{м}$.
За вычетом мебели не относящейся к рабочим местам (шкафы) на рабочие места
остаётся $\approx 20 \text{м}^2$. В помещении расположенно 2 рабочих места с ПК.
Следовательно на одного программиста приходится по $\approx 10 \text{м}^2$
площади помещения \lb и по $\approx 32 \text{м}^3$ объёма. Согласно \verb|СанПиН 2.2.2/2.4.1340-03| для
обеспечения оптимальных условий труда, на одного работника должно приходится не менее
$6 \text{м}^2$ площади и $19.5 \text{м}^3$ объёма. В данном помещении требования
СанПиН по размерам выполены.

\subsubsection{Микроклимат помещения}

Согласно \verb|ГОСТ 12.1.005-88| основными показателями,
характеризующими микроклимат являются:
\begin{mintemize}
\item температура воздуха
\item относительная влажность воздуха
\item скорость движения воздуха
\item интенсивность теплового излучения
\end{mintemize}
Все эти показатели влияют на теплообмен организма с окружающей средой.

Работа инженера-программиста относится к категории 1а, т.к. производится
сидя и сопровождается незначительным физическим напряжением. Уровень
энергозатрат не превышает $120 {\text{ккал}}/{\text{ч}}$ ($130 \text{Вт}$).

\newpage

Ниже приведена таблица с указанием оптимальных и фактических показателей,
характеризующих микроклимат помещения, для различных периодов года,
для категории работ "лёгкая-1а".

\begin{center}
\begin{tabular}{| c || C{2cm} | C{2cm} || C{2cm} | C{2cm} ||}
    \hline
    Параметр                   & \multicolumn{4}{c||}{Период года} \\
                                 \cline{2-5}
                               & \multicolumn{2}{c||}{холодный} & \multicolumn{2}{c||}{тёплый} \\
                                 \cline{2-5}
                               & опт.  & факт. & опт.  & факт. \\
    \hline
    Температура $^\circ C$     & 22-24 & 21-25 & 22-24 & 23-25 \\
    \hline
    Относительная влажность \% & 40-60 & 35-55 & 40-60 & 40-60 \\
    \hline
    Скорость движения воздуха  $\frac{\text{м}}{\text{с}}$ & 0.1 & 0.05-0.1 & 0.1-0.2 & 0.15-0.25 \\
    \hline

\end{tabular}
\end{center}

\subsubsection{Шум}

Основными источниками шума в помещении являются:
\begin{mintemize}
\item кондиционер ($25-30 \text{дБА}$)
\item компьютер ($30-50 \text{дБА}$ в зависимости от нагрузки)
\end{mintemize}

Согласно \verb|ГОСТ 12.1.003-83| уровни звукового давления в рабочем
помещении не должны превышать в октавных полосах со среднегеометрическими
частотами следующих значений, приведенных в таблице:

\begin{center}
\begin{tabular}{ | C{3cm} | *{8}{C{0.9cm}|} C{2cm} | }

\hline

Рабочее место & \multicolumn{8}{c|}{\begin{minipage}{10.4cm}Уровень звукового давления, дБ,\lb
               в октавных полосах со среднеквадратическими частотами, Гц\end{minipage}} & Уровни звука, дБА \\

\hline

\multirow{2}{*}{\begin{minipage}{2cm}\begin{center}Помещение\lbоператора\lbЭВМ\end{center}\end{minipage}} & 63 & 125 & 250 & 500 & 1000 & 2000 & 4000 & 8000 & \\

\cline{2-9}

 & 71 & 61 & 54 & 49 & 45 & 42 & 40 & 38 & 50 \\ [1cm]

\hline

\end{tabular}
\end{center}

При работе всего оборудования максимальный уровень шума составляет $80 \text{дБА}$, минимальный $55 \text{дБА}$, 
что не соответстует нормам. 

\newpage

\subsubsection{Электроопасность}

По степени поражения электрическим током помещение можно отнести к
помещениям без повышенной опасности, т.к. оно с нормальной температурой,
сухое, без пыли, пол паркетный (изолирующий), отсутствуют заземленные
металлоконструкции. Согласно \verb|ГОСТ 12.2.007.0-00| ПК можно отнести к
первому классу электротехнических изделий по способу защиты человека от
поражения электрическим током, т.к. ПЭВМ имеет провод с заземляющей
жилой и вилкой с заземляющим контактом для присоединения к источнику питания.
Основным источником опасности для людей, работающих в помещении,
представляет возможность поражения электрическим током при прикосновении
к корпусу ПК, при пробое изоляции плат на корпус. Предельно допустимые
значения напряжений прикосновения и токов при аварийном режиме
промышленных электроустановок напряжением до $1000 \text{В}$ и частотой $50 \text{Гц}$
должны соответствовать \verb|ГОСТ 12.1.038-82| и при продолжительности 
воздействия $1\text{с}$ не должны превышать $U=20\text{В}$ и $I=6\text{мА}$.

\subsubsection{Пожароопасность}
Согласно \verb|НПБ 105-03| помещение, в котором выполняется работа,
по пожарной опасности относится к классу В4. Источником возгораний
может стать замыкание в электропроводке. В помещении располагаются
возгораемые материалы: книги, бумага, пластмассовые конструкции --
всё это может стать причиной возникновения локальных возгораний. 

\subsubsection{Освещённость}
Работу инженера за компьютером можно отнести к работе со средней
точностью (наименьший размер объекта различения от 0.5 до 1мм)
IV-го разряда зрительной работы, с большой контрастностью объекта
различения на светлом фоне (например, символов на экране дисплея),
подразряд зрительной работы Г. Помещение для размещения операторов
можно отнести к 1-ой группе помещений, в которых производится
различение объектов зрительной работы при фиксированном направлении
линии зрения работающего на рабочую поверхность. Для такого разряда
зрительной работы нормируемое значение коэффициента естественной
освещенности (КЕО) (при совмещенном освещении) должно соответствовать
\verb|СниП 23-05-95| и составлять не менее 4\%.
Естественное боковое освещение помещения осуществляется с
помощью одного углового окона размерами 100 и 45 см по ширине на 210 см по высоте.
Искусственная освещенность осуществляется с помощью комбинированной системы
освещения с использованием люминесцентных ламп. Величина искусственной
освещенности при систематическом использовании дисплеев и вводе
данных в ЭВМ должна быть не ниже 200 лк.
При искусственном освещении источниками света являются встроенные
в подвисном потолке светильники в количестве 15 штук по 1 светодиодной
лампе в каждом. Каждая лампа потребляет 12Вт и испускает поток в 950лм.
Т.о. можно определить величину искусственной освещенности:

$$ E_\text{иск} = \frac{\text{Ф}_\text{общ}}{S} = \frac{n \cdot \text{Ф}}{S} =
\frac{15 \cdot 960}{27.95} \approx 515 \text{ лк } $$

где: 
\begin{mintemize}
    \item $n = 15$ - количество источников освещения
    \item $S = 27.95 \text{м}^2$ - площадь помещения
    \item $\text{Ф} = 950 \text{лм}$ - световой поток
\end{mintemize}

Это удовлетворяет \verb|СНиП 23-05-95|, где величина искусственной
освещенности при системе общего освещения должна быть не меньше 500 Лк.
Световой поток распределяется по комнате равномерно. Экраны мониторов
оснащены антибликовым покрытием. Соотношение освещенностей рабочей
поверхности и окружающего пространства для работ с мониторами ВДТ
не должно быть больше чем 3:1. Это требование в данном помещении выполняется.
