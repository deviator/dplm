\subsection{Заключение}

В разделе были рассмотрены следующие вопросы: 
\begin{mintemize}
\item Характеристики помещения и организация рабочего места
\item Микроклиматические нормы для данного класса помещения
\item Источники шума и их интенсивность
\item Электроопасность при работе с ПК
\item Пожароопасность
\item Освещённость рабочего места
\end{mintemize}

Результаты исследования: 
\begin{mintemize}
\item Площадь и объём помещения позволяют свободно разместить два
    рабочих места. Каждое из рабочих мест эргономично укомплектовано
    в соответствии необходимым стандартам.
\item Ответственность за поддержание комфортной температуры возложена на
    кондиционер. Расчёт мощности кондиционера производился исходя из
    объёма помещения с учётом рабочего оборудования. Кондиционера хватает
    с запасом.
\item Шум издают в основном только 3 устройства: 2 ПК и 1 кондиционер.
    Остальные шумы не постоянны или слабы (шум улицы через стеклопакет)
    и могут не учитываться в расчёте. В помещении приемлемый уровень шума.
\item Была проведена проверка заземления розеток. Розетки действительно
    заземлены и следовательно оборудование, подключаемое через используемые
    удлинители так же заземлено.
\item Локальными причиными возникновения пожара могут быть только короткие
    замыкания проводки. На случай возгараний предусмотрены пожарная сигнализация
    и огнетушитель.
\item Естественного освещения не достаточно для комфортной работы даже днём.
    Искусственное освещение позволяет комфортно работать в любое время суток.
    Используемые диодные лампы имеют цветовую температуру 3000К, тоесть мягкий
    желтоватый цвет. Это немного смягчает утомляемость глаз при работе с
    искусственным освещением.
\end{mintemize}
