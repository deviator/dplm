\newpage
\section{Написание симулятора}

Основой частью работы можно считать программное обеспечение для моделирования
массива. Главными проблемами являлись построение карты и отображение процесса работы системы
в реальном времени. Решением этих двух задач является использование гетерогенных 
вычислений с интероперабельностью с графикой. Были применены распостранённые 
технологии такие как \verb|OpenGL| и \verb|OpenCL|.

\section{Модель карты местности}

Предполагается что участок местности ограничен по ширине, глубине и высоте.
Он разбивается на прямоугольные сектора, каждый из которых хранит информацию об единице объёма местности:
\begin{mintemize}
    \item был ли сектор исследован
    \item содержится ли в секторе что-либо
\end{mintemize}

Реализация карты представляет собой 1-мерный массив, к которому можно обратиться с помощью трёх
индексов-координат. Так же для карты имеется матрица трансформации координат из локальных (индексов)
в глобальные (метры).

\section{Модель единицы массива}

За основу была взята концепция БПЛА с вертикальным взлётом
(коптер или вертолёт), тоесть юнит имеет возможность зависать в точке и резко изменять направление движения.
Назовём единицу массива юнитом. Предполагается что каждый юнит имеет возможность
оценить дальность в любом направлении, в нескольких точках одновременно с определённым 
угловым разрешением (карта глубин). Также предполагается, что каждый юнит знает точно положение и скорость
свои и остальных юнитов в массиве.

\section{Заполнение карты}

Каждый юнит получает информацю о мире с помощью датчика глубины в виде картинки,
где каждому пикселю соответствует измерение дальности в определённых угловых координатах. В работе не эмулировлись
дистрозийные искажения, поэтому для приведения карты глубин в точки в системе координат юнита достаточно 
матрицы перспективной трансформации, которая имеется у каждого юнита. Она так же может меняться в процессе работы
системы при необходимости (изменение угла обзора, зум). После получения точек в связанной системе координат они приводятся
сначала к мировой, затем к системе координат карты. Так же приводится положение юнита (камеры) к системе координат карты.
Заполняется карта по следующему алгоритму: строится отрезок из точки камеры к точке, полученной с датчика, все сектора, которые
находятся между помечаются как исследованные, сектор, в котором находится точка с датчика помечается как заполненая.

\section{Логика перемещения юнитов}

Для системы в целом ставится задача полностью исследовать объём, отражаемый в карте.
Из этого следует что каждый юнит направляется к ближайшему неизведанному участку. 
На рисунке можно видеть жёлтые области -- эти области не обошёл сенсором ни один юнит.

\section{Алгоритм перемещения юнитов}

Важно чтобы юниты не сталкивались друг с другом и не врезались в стены.
Для этого предусмотрены 2 алгоритма коррекции направления движения юнита: 
\begin{mintemize}
    \item каждый юнит проверяет все остальные на предмет сближения и пересечения моментальной траектории
            (луч из юнита в направлении перемещения)
    \item происходит обращение к карте в целях получения ближайших заполненных секторов
\end{mintemize}
