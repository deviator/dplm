\newpage
\section{Введение}

Всё чаще возникает проблема ведения боевых действий
в черте города. В связи с этим растёт необходимость в технических
решениях, которые позволят быстро и безопасно разведать местность.
Массив миниатюрных и дешёвых БПЛА может справиться с этой задачей 
лучше, как никто другой. Также такой массив может решать задачи недоступные
для решения с помощью классических беспилотников для наблюдения, а именно: 
\begin{mintemize}
    \item построение трёхмерной динамической карты местности
    \item исследование помещений
    \item непрерывный трекинг подвижных объектов
\end{mintemize}

Также к плюсам такой системы можно отнести высокую робастность в плане
устойчивости к потерям единиц, так как количество единиц в массиве
практически не ограниченно, а потеря даже 90\% массива не сможет
вывести систему из строя до конца.  Качество решаемой задачи
будет плавно падать с падением количества единиц массива.

В понятие информационного поля местности также может входить
и возможность доступа оператора к данным датчиков любого из юнитов, например,
к камере, работающей в оптическом диапазоне.
